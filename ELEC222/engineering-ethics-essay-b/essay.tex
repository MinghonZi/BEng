\documentclass[12pt, a4paper]{article}
\title{An Essay on ELEC222 Engineering Ethics B}
%\author{Minghong Xu}

\usepackage{tocloft}
\renewcommand{\thesection}{\Roman{section}}
\addtolength{\cftsecnumwidth}{0.5em}

\setlength{\parindent}{0pt}
\setlength{\parskip}{9pt plus 2pt minus 2pt}

\begin{document}
\maketitle
\tableofcontents

\section{Scenario}
An aircraft of a small airline crashed, becuase the studs of a pump casing had failed as per a report from one of the maintenance team's engineer. The Head Office requires all maintenance teams to replace all pump casing tuds on every jet of this kind in the fleet quickly.

However, the engineer who worte the crash report examined the new studs and found that they are not physically strong enough to ensure the safety of the pump casing. He concluded that this may cause further accidents.

Under normal circumstances, he has obligation to obey this kind of urgent orders. On the other hand, he is hoping for promotion in the next six months. He does not want to jeopardise his future by causing problems for the company in this matter.

\section{Dilemma}
The main contradiction is that, with the urgent order to replace the studs, the action to ensure the safety of aircrafts conflicts with the order. To take steps to ensure the safety would contrary to the order, which offenses an airline employee's obligation. Meanwhile, disobeying the order could also threaten future career advancement. On the other hand, in order to fulfil the employee's obligation and also for sake of the career, simply following the order and compromising the aircrafts' safety is considered as an irresponsible and slapdash behaviour. Furthermore, other engineers may find that safety issue has been deliberately igonored, thus affecting the reputation as an engineer.

\section{Three Options}
\begin{description}
  \item[option one] Comply with the empolyee's obligation; obey the order and compromise safety.
  \item[option two] Disobey the order to ensure the safety of aircrafts.
  \item[option three] Spent some time informing the superior or using other appropriate channels.
\end{description}

\section{Discussion: logic for best option}
The first option does not sufficiently consider the consequences. If a second accident occurs because of the same studs failure, the airline will lose public trust, lose customers, and eventually start laying off employees due to operational difficulties. The engineers who are directly reponsible for the safety of aircrafts as maintenance staff are likely to be the first to be laid off. The first option has the aim of promotion, but may ultimately have the opposite of his wish. More importantly, the first option against the first ethical principle listed in Royal Academy of Engineering's statement: honesty and integrity \cite{statement-of-ethical-principles}. In the fisrt principle it writes: declare conflicts of interest and avoid deception and take steps to prevent or report corrupt practice of professional misconduct. This option is deception since it is hiding the information which could lead to serious accidents. In addition, if other engineers find the safety issue is ignored deliberately, it can diminish the reputation of the engineer, which breaks the last point in the second principle: uphold the reputation and standing of profession \cite{statement-of-ethical-principles}.

The second option is also not ideal. As a whole, one person refusing to comply with the order does not allow the entire maintenance team to make any effective changes in such an emergency. Addressing the safety issue required reporcurement of studs, replacement, and testing. This is a huge project, with multiple steps to complete. One engineer is not enough to do it all in short time. What is needed in this case is an approach that would have an impact on the whole orgnisation, ranther than individuals refusing to collaborate. Moreover, this inappropriate, non-holistic approach is likely to have a negative impact on the future promotion. Therefore, it is an act that neither solves the issue effectively nor protects one's interests.

The third option should be the best among the three. This approach tries to solve the issue at the organisational level. However, the superior might ignore the concern. In this case, it is still possible to go directly to senior officers to report the risk. The whole point for the order is to ensure the fleet safety, so this should work. For any reason, if neither of these take effect, one should turn to other appropriate channels. As a counterexample, media is not a acceptable channel. Engineering professionals shall act in a reliable and trustworthy manner as per the first point in the fisrt principle, but put concern on the media will increase the public worries and lost the trust \cite{statement-of-ethical-principles}. Furthermore, put such information on the media offense the airline's confidentiality. This action violates the obligation as an employee and also breaks the third point in the first principle \cite{statement-of-ethical-principles}. One possible appropriate channel might be the outside inspection agencies. An anonymous letter could be given to describle the situation.

Engineering professionals have a duty to "identify, evaluate, quantify, mitigate and manage risks" \cite{statement-of-ethical-principles}. To put this principle into practice and also to comply with other principles, as well as to protect one's own interests to a certain extent, the third option is the best among the three option to take.

\bibliographystyle{IEEEtran}  
\bibliography{refs.bib}
\addcontentsline{toc}{section}{References}

\end{document}