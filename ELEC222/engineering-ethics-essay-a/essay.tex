\documentclass[12pt, a4paper]{article}
\title{An Essay on ELEC222 Engineering Ethics A}
%\author{Minghong Xu}

\begin{document}
\maketitle
\tableofcontents

\section{Example \MakeUppercase{\romannumeral 2}}
\subsection*{Dilemma}
You are presented with what looks like an excellent opportunity to setup a prosperous mining operation and also to provide some benefit to a local population.

However, in order to get this operation off the ground, it looks like you may have a bypass certain legal channels, and to perform a quid pro quo service for the Mayor which could be interpreted as a bribe.

\subsection*{Analysis/Discussion}
In this scenario, it is crucial that everything is legal. If a mining licence obtained by bypassing the legality check proves to be illegal, firstly it could affect the company's reputation, and secondly the mining activity may be immediately terminated. This would not only lead to a loss of investment, but would also reduce the trust of shareholders. In addition, the absence of a legality check means that mining activities may have a negative impact on the local environment beyond what is tolerated. Therefore, bypassing legal channels to obtain the mining license means disrespecting the law and the public good.

Assuming that the company is highly ethical, acceptance of the terms offered by the mayor depends on the mayor's ability to ensure that both the construction of the hospital and the qualification for mining are perfectly legal. However, the ambiguous answer given by the mayor is highly questionable. He did not guarantee procedural legality, but ranther the opportunity to mine in the area.

\subsection*{Recommendation}
It would be a shame to abandon mining in an area that is likely to be rich in gold. While nogotiating with the mayor to see if he can ensure that everything can be done legally, find other ways to obtain the license.

\section{Example \MakeUppercase{\romannumeral 3}}
\subsection*{Dilemma}
You have started consultancy work on a project to develop a sophisticated monitoring system for a residential building, and you learn that the proposed use is as a surveillance system for elderly and infirm individuals.

There is a concern that as some of those individuals will not have the mental capacity to understand the system, and so will not be able to fully consent to the system.

Your work might lead to an invasion of individuals' privacy.

\subsection*{Analysis/Discussion}
The monitoring system is intended to help even frail elderly people. However, the system has the potential to invade the privacy of the end user. It is disrespectful for the engineer responsible for designing the system to ignore the potential hazards to the user.

Engineers, as a social role, are trusted and expected by society. When people expect that the systems they will use will not violate privacy and trust its desinger can fix this flaw, an engineer who has good ethics should meet this expection. If necessary, the whole system can even be discarded.

\subsection*{Recommendation}
If the whole monitoring system is designed in such a way that it is diffcult to circumvent the problem of invasion of privacy, the engnieer can decide to quit from the project.

If there is an apporach to ensure the user privacy is protected well, then the engnieer shall conduct that apporach and continue developing the system.

\end{document}